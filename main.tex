\documentclass[xcolor=table]{beamer}
\mode<presentation>
{
  \usetheme{Madrid}      
  \usecolortheme{default} 
  \usefonttheme{default}  
  \setbeamertemplate{navigation symbols}{}
  \setbeamertemplate{caption}[numbered]
} 

\usepackage[english, russian]{babel}
\usepackage[utf8x]{inputenc}

\title{Методика многокритериальной оптимизации экономических решений}
\author{Курсовая работа Писовой Екатерины}
\institute{Научный руководитель: доцент, кандидат ф.-м. наук, Ефимова Г.А.}
\date{Одесса - 2018}

\begin{document}
\begin{frame}
  \titlepage
\end{frame}
\section{Цель работы}
\begin{frame}{Цель работы}
\begin{itemize}
  \item Рассмотрение экономических ситуаций, когда возникает задача многокритериальной оптимизации.
  \item Рассмотрение методов многокритериальной оптимизации.
  \item Применение некоторых методов многокритериальной оптимизации для решения конкретной задачи и сравнение полученных результатов.
\end{itemize}
\end{frame}

\section {Пример}
\begin{frame}{Пример}
\begin{table}[]
\begin{tabular}{|l|l|l|l|l|}
\hline
 & WW   & Opel  & Ford  & Toyota \\ \hline
\begin{tabular}[c]{@{}l@{}}
Цена \\ (1000 Euro)\end{tabular}           
& 16.2 & 14.9 & \cellcolor[HTML]{68CBD0} 14.0 & 15.2 \\ \hline
\begin{tabular}[c]{@{}l@{}}
Расход топлива\\ (на 100 км)\end{tabular} 
& 7.2  & \cellcolor[HTML]{68CBD0}7.0 & 7.5  & 8.2  \\ \hline
Мощность & 66.0 & 62.0 & 55  & \cellcolor[HTML]{FD6864}71 \\ \hline
\end{tabular}
\end{table}
\par \hspace{0.7cm} Какой автомобиль нужно выбрать, чтобы он был мощным, \\ \hspace{0.7cm}недорогим и с малым расходом топлива?
\end{frame}

\section{Общий вид многокритериальной задачи}
\begin{frame}{Общий вид многокритериальной задачи}
$$
\left.\begin{matrix}
y_{k} = f_{k}\left ( x \right )\rightarrow opt, k =\overline{1,p}
\\x\epsilon  X
\end{matrix}\right\}$$
\begin{itemize}
\item[] $p$ - количество целевых функций, которые подлежат оптимизации,
\item[] $f_{k}\left ( x \right )$ - отдельная $k$-тая функция из критериального набора,
\item[] $X$ - множество допустимых планов.
\end{itemize}
\end{frame}


\section{Задача}
\begin{frame}{Задача}
\hspace{0.7cm} Рассмотрим двокритериальную задачу оптимизации. Её математическая модель имеет вид:
$$
\left.\begin{matrix}
y_{1} = 5x_{1} + 9x_{2} \rightarrow max
\\ y_{2} = 40x_{1} + 8x_{2} \rightarrow max
\\ 7x_{1} + 8x_{2}\leq 56
\\5x_{1} + 9x_{2} \leq 80
\\x_{1},x_{2}\geq 0
\end{matrix}\right\}
$$
\end{frame}

\subsection{Решение}
\begin{frame}{Решение}
\begin{itemize}
\item Оптимальное решение методом <<идеальной>> точки: $y_{1} = 40, y_{2} = 320$ достигается при $x_{1} = 8, x_{2} = 0$.
\item Оптимальное решение методом последовательных уступок:  $y_{1} = 40.3511, y_{2} = 318$ достигается при $x_{1} = 7.8779, x_{2} = 0.1068$.
\item Оптимальное решение с помощью обобщенной методики многокритериальной оптимизации: $y_{1} = 40, y_{2} = 320$ достигается при $x_{1} = 8, x_{2} = 0$.
\end{itemize}
\end{frame}
\frame{
    \frametitle{Конец доклада}
    \centering
    {
     Благодарю за внимание!
    }
}
\end{document}
